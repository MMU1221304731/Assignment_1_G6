\documentclass[a4paper, 12pt]{article}
\usepackage{apacite}
\usepackage[T1]{fontenc}
\usepackage[utf8]{inputenc}
\usepackage{mathptmx}
\usepackage{enumerate}
\usepackage[margin=0.5in]{geometry}
\usepackage{lipsum}

\renewcommand{\baselinestretch}{1.0}

\newcommand\nd{\textsuperscript{nd}\xspace}
\newcommand\rd{\textsuperscript{rd}\xspace}
\newcommand\nth{\textsuperscript{th}\xspace} %\th is taken already

\setlength\parindent{0pt} % set paragraph indent to zero

% fill up your name, ID and paper title here
\author{
LOA TAT ANN \quad 1221304731 \quad  contribution1 \\
YU BUI XUAN \quad 241UC24121\quad contribution2\\
MOHAMAD SYAMEL BIN MOHAMAD KARID \quad 1221309130 \quad contribution3\\
% Student Name4 \quad Student ID4 \quad contribution4\\
}
\title{Comparison of Machine Learning Methods Used for Detecting DDoS Attacks in Performance and Accuracy}

\begin{document}
\maketitle


\begin{abstract}

\end{abstract}

\section{Introduction and Problem Statement}

\subsection{Introduction}
As the world has become increasingly connected, the rapid evolution of network technologies has also led to cyberattacks becoming much more common and harder to contain. This includes DDoS (Distributed Denial of Service) attacks, which try to disrupt network services by overloading a network with malicious traffic (bots) and denying normal system access. \citeA{1} Such attacks can even affect and suffer the largest companies such as the online retailer and cloud service provider Amazon in 2020, where a DDoS attack was detected on its services with a recorded access rate of 2.3 Tbps, and in 2016, where a DDoS attack disrupted a major DNS infrastructure provider which caused several major networks like Twitter (X) and PayPal became inaccessible to users for 3 hours. \citeA{1} 

There are several ways DDoS attacks the host system, including sending malicious traffic at a slow pace to a target (low-rate attacks), using a high amount of packets to compromise the system (high-rate) attacks, exploiting network protocol vulnerabilities to use up resources (protocol exploitation), and substituting the IP address of the victim with the attacker's (reflection-amplification). \citeA{1} 

To prevent such attacks, various methods are also developed to detect, prevent and mitigate such attacks from harming the system. One such example involves machine learning as they have good performance on network anomaly detection, and it can be adapted to deal with new threats. \citeA{3}

\subsection{Problem Statement}
Researchers use several different test methods to detect DDoS attacks with machine learning. The main methods are supervised and unsupervised methods. Supervised methods involve a physical network in a lab-based environment where the attacker and the victim exist in the same network and all tests are controlled scientifically. On the flip side, unsupervised methods test the system with real-world networks. The data collected from such methods are analyzed based on their characteristics. \citeA{1}

With the development of various methods available, it can be difficult to determine the best method for detecting DDoS networks for a system, as they have different needs and constraints. Furthermore, each detection scenario is different, as they do not always have the same situation and attributes as others. 

This research aims to compare various machine learning models by comparing their accuracy, performance and efficiency of the models regarding detecting DDoS attacks, and determine which methods are suitable for different environments. 

\section{Methodology}

\subsection{Methods used for comparison}

\subsubsection{TPAAD}
The TPAAD two-phase authentication system, written by Najmun Nisa, Adnan Shahid Khan, Zeeshan Ahmad, and Johari Abdullah, explains a new method that utilizes machine learning algorithms such as Support Vector Machine (SVM) and K-Nearest Neighbors (KNN) to detect DDoS attacks on Software-Defined Networks (SDNs) efficiently. The system also filters incoming packets to detect and stop malicious traffic without host deactivation and affecting normal network connectivity. \citeA{3} The detection for the system works by utilizing two different modules, which are for detecting threats and mitigating the risks. 

The detection module continuously monitors the network to identify abnormal traffic. Then packet filtration techniques used by the mitigation module are used to split network traffic to contain and find suspicious activity with machine learning models referring to a predefined dataset (CICDoS 2017). After that, the system employs a tunnelling mechanism to block or reroute malicious traffic ensuring normal network activity is unaffected. \citeA{3}

\subsubsection{Flexible SDN-Based Architecture}

Low-rate DDoS attacks are more challenging to contain as they target network protocols such as TCP without overloading the system. Because of this, they are harder to detect as they are "integrated" with legitimate network traffic and are difficult to detect and report, and data for such attacks are harder to extract. Hence, the paper proposes a novel modular system architecture to detect such incidents with machine learning techniques. \citeA{4}

The system detects such threats by utilizing two

\subsubsection{ML-DDoSnet}

IoT \cite{5}

\section{Results}

\subsection{TPAAD}
According to the tests, the TPAAD system performs marvellously when it comes to the detection of DDoS attacks. The system recorded a 99.56\% accuracy on detecting DDoS attacks when testing with the CICDoS 2017 dataset on a controlled environment running on an Ubuntu VM on VMWare and an SDN setup created via Mininet and the RYU controller. \citeA{3}

\section{Discussions}
\subsection{Overview Of The Methods}
\subsection{Analysis of Individual Methods}
\subsubsection{TPAAD ???}
\subsubsection{Flexible SDN-Based Architecture ??}
\subsubsection{ML-DDoSnet}
\subsubsection{errr continuee}
\subsection{Contribution Of The Research}

\section{Conclusion}
% Discussions include but not limited to:
%\begin{enumerate}[(a)]
%\item Pros and cons of the method/algorithm used,
%\item limitation of the method/algorithm,
%\item contribution of the research.
% \end{enumerate}
%Conclusion of the work presented in the reviewed research papers.

There are some pros and cons of such methods. For example, The TPAAD system records good accuracy in detecting threats while using minimal resources through efficient network filtering and traffic management. Also, the system can dynamically adjust to other threats to reduce disruptions. However, the system has the potential for delays due to the processing of traffic, as well as a highly complex system which increases the difficulty of deployment. Besides that, the system's reliability is highly dependent on dataset quality which could increase the false negatives of the system if an inadequate dataset is used. \citeA{3}

Every problem must have a solution, but the best solution depends on the use case and other factors. DDoS attacks are still a major concern for Internet network services, and the challenge remains. Despite this, machine learning will still need more work to have the best results in detecting threats, in a balanced approach that combines cost and performance equally. As the need for cybersecurity and machine learning evolves, the collaboration of all parties involved will be crucial to protecting our networks from DDoS attacks. 

\section{Future work}
Despite the work, the report still faces some issues. One suggestion for the report is 

%References
\bibliographystyle{apacite}
\bibliography{MyBib}{}


\end{document}

