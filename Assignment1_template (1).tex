\documentclass[a4paper, 12pt]{article}
\usepackage{apacite}
\usepackage[T1]{fontenc}
\usepackage[utf8]{inputenc}
\usepackage{mathptmx}
\usepackage{enumerate}
\usepackage[margin=0.5in]{geometry}


\renewcommand{\baselinestretch}{1.0}

\newcommand\nd{\textsuperscript{nd}\xspace}
\newcommand\rd{\textsuperscript{rd}\xspace}
\newcommand\nth{\textsuperscript{th}\xspace} %\th is taken already

\setlength\parindent{0pt} % set paragraph indent to zero

% fill up your name, ID and paper title here
\author{
LOA TAT ANN \quad 1221304731 \quad  contribution1 \\
YU BUI XUAN \quad 241UC24121\quad contribution2\\
MOHAMAD SYAMEL BIN MOHAMAD KARID \quad 1221309130 \quad contribution3\\
% Student Name4 \quad Student ID4 \quad contribution4\\
}
\title{Comparison of Machine Learning Methods Used for Detecting DDoS Attacks }

\begin{document}
\maketitle

\begin{abstract}
     This report provides a comprehensive review about the various ways of machine learning techniques that could be used to detect DDoS attacks. Besides that, we will also find important data to determine which method is the most effective against DDoS attacks and how does it help against the threat. 
\end{abstract}

\section{Introduction and Problem Statement}

\subsection{Introduction}
As the world has becoming increasingly connected, the rapid evolution of network technologies have also led to cyberattacks have becoming much more common and harder to contain. One of the major threats among them are DDoS (Distributed Denial of Service) attacks, which is a method of cyberattack that affects network services by overloading a network with malicious traffic.\cite{4} Such attacks have caused major losses to even the largest companies such as Amazon in 2020. 

\subsection{Problem Statement}
DDoS attacks are currently a major threat to the usability of network infrastructure. There are several ways DDoS attacks the host system, including sending malicious traffic at a slow pace to a target (low-rate attacks), using high amount of packets to compromise the system (high-rate) attacks, and exploiting network protocol vulnerabilities (protocol exploitation). \cite{1} The methods uses different ways to disrupt and affect usability of the network causing users unable to access the network normally, and creating monetary losses to the company. To prevent from such attacks, various methods are also developed to detect, prevent and mitigate such attacks from harming the system. However, the capabilities for detecting such attacks are not sufficiently advanced enough to effectively prevent such issues, due to a lack of funding. \cite{1} Despite this, new methods are developed and used to deal with this situations, which includes machine learning methods. Despite the difficulties faced on its implementation, machine learning methods is still widely used due to its strong solving performance. \cite{7}

\section{Methodology}

Currently there are several methods where machine learning can be implemented to detect DDoS attacks. One such implementation is a two phase authentication system to precisely detect and mitigate malicious network activities. There are two phases for this system to work properly. The first phase includes the use of packet filtration and classification techniques to detect incoming network packets in search for dangerous packets based on features with the use of machine learning algorithms such as Support Vector Machines (SVM) and K-Nearest Neighbours (KNN) with use of the CICDoS 2017 dataset for reference. Following that, the second phase involves a targeted restriction of the network, which it deploys tactics such as rate limiting, IP blocking and traffic redirection to stop harmful network activity caused by an DDoS attack from affecting the host system, without the need for an deactivation of the network. The two phase system will reduce the risk for DDoS attacks while maintaining the system's normal operation. Additionally, machine learning models can use the data to improve its detection systems by "learning" the data and adapt with it, which increases performance and accuracy over time to combat against evolving threats. 

\cite{3} 

Besides that, another method for this process includes the use of a SDN architecture to identify SDN based \cite{5} 

\section{Result}

\subsection{TPAAD}
According to the tests, the TPAAD system performs marvellously when it comes to the detection of the DDoS attacks. The system recorded a 99.56\% attack detection accuracy when testing with the CICDoS 2017 dataset on a controlled environment running on a Ubuntu VM on VMWare and a SDN setup created via Mininet and the RYU controller. \cite{3}

\section{Discussions}
Discussion in a new paragraph.

\section{Conclusion}
% Discussions include but not limited to:
%\begin{enumerate}[(a)]
%\item Pros and cons of the method/algorithm used,
%\item limitation of the method/algorithm,
%\item contribution of the research.
% \end{enumerate}
%Conclusion of the work presented in the reviewed research papers.

As of today, DDoS attacks are still a major concern for Internet network services, and the challenge of detection still remains. As such, this paper provides the information on how machine learning work, and the comparison between them to see which are the better choices for the detection of DDoS networks. 

There are some pros and cons of such methods used. For example, The TPAAD system has a good accuracy on detecting threats while using a low amount of resources by efficient network filtering and traffic management. Also, the system can dynamically adjust to other threats to reduce disruptions. However, the system has a potential for delays due to the processing of traffic, as well as a highly complex system which increases the difficulty of deployment. Besides that, the system's reliability is highly dependant on dataset quality which could increase the false negatives of the system if an inadequate dataset is used. \cite{3}

Overall, its important that every problem has a solution, but the best solution depends on use case and other factors. Despite this, the integration of machine learning for such purposes will still need more work in order to have the best results, in a balanced approach that combines cost and performance in a equal way. As the field of cybersecurity and machine learning evolves, the collaboration of all parties involved will be crucial to maintaining our networks from DDoS attacks.  

\section{Future work}
Despite the work there are still some issues while facing the report. One suggestion for the report is 

%References
\bibliographystyle{apacite}
\bibliography{MyBib}{}


\end{document}

